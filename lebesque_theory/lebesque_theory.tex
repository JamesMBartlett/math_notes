\documentclass{article}
\usepackage[utf8]{inputenc}
\usepackage[english]{babel}
\usepackage{amsmath}
\usepackage{amsthm}
\usepackage{amssymb}
\usepackage{mathabx}
\usepackage{mathtools}
\theoremstyle{definition}
\newtheorem{definition}{Definition}[section]

\newtheorem{prop}{Proposition}[section]
\newtheorem{lemma}{Lemma}[section]
\newtheorem{example}{Example}[section]

\newtheorem{theorem}{Theorem}[section]
\newtheorem{corollary}{Corollary}[theorem]
\theoremstyle{remark}
\newtheorem*{remark}{Remark}
\theoremstyle{remark}
\newtheorem*{notation}{Notation}

\def\reals{{\mathbb R}}
\def\scriptm{{\mathcal M}}
\def\integers{{\mathbb Z}}
\def\naturals{{\mathbb N}}
\def\rationals{{\mathbb Q}}

\title{105 Measure Theory Notes}


\begin{document}
\section{Measurable Functions}
We want to use this improved notion of measurability to define an improved integral (the ''lebesque integral''). We must first discuss the types of functions that we will attempt to integrate.
\begin{definition}
A function $f: \reals^n \to \reals$ is Lebesque measurable if for all open sets $U \subseteq \reals$, we have $f^{-1}(U) \in \scriptm(\reals^n)$
\end{definition}

Since open sets are measurable, we immediately see that $f: \reals^2 \to \reals$ continuous is lebesque measurable. We can say more:
\begin{prop}
If $f: \reals^n \to \reals$ is Riemann integrable than it is lebesque measurable.
\end{prop}
\begin{proof}
Let $D\in \reals^n$ be the discontinuity set of $f$. By the Riemann-Lebesque Theorem, $D$ is a zero set. Hence $D \in \scriptm$ with $m(D) = 0$. Define a continuous $g : D^c \to \reals$ by $g(x) = f(x)$. Then for any $S \subseteq \reals$, it is easy to see that $g^{-1}(S) \subseteq f^{-1}(S) \subseteq g^{-1} \cup D$. In particular, if $S$ is open then $g$ continuous implies $g^{-1}(S)$ is open and hence measurable. But then $f^{-1}(S) = g^{-1}(S) \cup (D \setminus f^{-1}(S))$ which are both measurable so $f^{-1}(S) \in \scriptm$ and so $f$ is lebesque measurable.
\end{proof}

\begin{prop}
Let $f: \reals^n \to \reals$. The following are equivalent: \begin{itemize}
	\item $f$ is lebesque measurable
	\item For all closed subsets $V \subseteq \reals$, $f^{-1}(V) \in \scriptm$
	\item For all $a \in \reals$, $f^{-1}((a, \infty))\in \scriptm$
	\item For all $a\in\reals$, $f^{-1}((-\infty, a))\in\scriptm$
	\item For all $a\in\reals$, $f^{-1}([a, \infty))\in \scriptm$
	\item For all $a\in\reals$, $f^{-1}((-\infty, a])\in\scriptm$
\end{itemize}
\end{prop}

\begin{prop}
Let $f,g: \reals^n \to \reals$ be measurable. Then $f + g$ and $f \circ g$ are measurable.
\end{prop}
\begin{proof}
Let $a \in \reals$. By the previous prop, it suffices to show $$(f + g)^{-1}([a, \infty)) \in\scriptm$$
Observe that, $$(f + g)(x) \geq a \Leftrightarrow g(x) \geq a - f(x)$$.
We claim $$ (f + g)^{-1} ([a, \infty)) = \cap_{n\in\naturals} \cup_{k\in\integers} f^{-1}((-\infty, \frac{k}{n}]) \cap g^{-1}([a - \frac{k}{n}, \infty))$$
For $x$ in RHS, $$f(x) + g(x) \geq a \Rightarrow g(x) \geq a - f(x)$$. 
Then $\forall n \in \naturals, \exists k \in \integers$ such that $f(x) \leq \frac{k}{n}$. Hence, $g(x) \geq a - \frac{k}{n}$. So $$x\in \cup_{k\in\integers} f^{-1}((-\infty, \frac{k}{n}]) \cap g^{-1}([a - \frac{k}{n}, \infty))$$. 
Since this holds for all $n\in\naturals$, $x$ is in LHS.
Conversely, let $x$ in LHS, Then $\forall n\in\naturals$, $\exists k \in \integers$ such that $\frac{k-1}{n} \leq f(x) \leq \frac{k}{n}$. Since $f(x) \leq \frac{k}{n} \Rightarrow g(x) \geq a - \frac{k}{n}$. Thus $f(x) + g(x) \geq \frac{k-1}{n} + a - \frac{k}{n} = a - \frac{1}{n}$. Letting $n\to\infty$ yields $f(x) + g(x) \geq a$. So $x$ in RHS. Now since $f$ and $g$ are measurable, the above proposition implies that $f^{-1}((-\infty, \frac{k}{n}]) \cap g^{-1}([a - \frac{k}{n}, \infty))$ is measurable. Since $\scriptm$ is closed under countable unions and intersections, we obtain $(f + g)^{-1}([a, \infty))\in \scriptm$. Thus $(f + g)$ is lebesque measurable. The proof for $f \circ g$ is similar but now one must be careful about the signs of $a, \frac{n}{k}$.   
\end{proof}
\begin{prop}
Let $(f_k)_{k\in\naturals}$ be a sequence of uniformly bounded measurable functions from $\reals^n \to \reals$. Then $g_1(x) = \sup_k f_k(x)$, $g_2(x) = \inf_k f_k(x)$, $g_3(x) = \limsup_{k\to\infty} f_k(x)$, $g_4(x) = \liminf_{k\to\infty} f_k(x)$, are all measurable and if the sequence converges pointwise, then $\lim_{k\to\infty} f_k$ is measurable.
\end{prop}
\begin{proof}
Observe that $\forall a \in \reals$, 
\begin{align*}
g_1^{-1}((a, \infty)) &= \cup_{k\in\naturals} f_k^{-1}((a, \infty)) \in \scriptm \\
g_2^{-1}((-\infty, a)) &= \cup_{k\in\naturals} f_k^{-1}((-\infty, a)) \in \scriptm \\
\end{align*}
Thus $g_1$ and $g_2$ are measurable. But then $$g_3(x) = \inf_N(\sup_{k\geq N} f_k(x))$$ is measurable, and similarly for $g_4$. Finally, if the sequence converges pointwise then $\lim_{k\to\infty} f_k = g_3 = g_4$ so it is measurable.
\end{proof}
\begin{corollary}
If $f,g: \reals^n \to \reals$ are measurable, then so are $\max(f, g)$ and $\min(f, g)$.
\end{corollary}

Recall that for the Riemann integral. The fundamental functions that all Riemann integrable function were approximated by were step functions. The big diference between the Riemann integral and the ``Lebesque integral'' will come from swapping out step functions for ``simple functions''. 
\begin{definition}
A function $f: \reals^n \to \reals$ is called a simple function if $\exists c_1, \dots, c_d \in \reals$ and $E_1, \dots, E_d \in \scriptm$ such that $$f = \sum_{i=1}^d c_i \chi_{E_i}$$.

That is, a simple function is a finite, linear combination of characteristic functions of measurable subsets.
\end{definition}

Observe that for $a\in \reals$, and $f$ a simple function as above $$f^{-1}([a, \infty)) = \begin{cases} 
\cup_{c_i \geq a} E_i & \text{if } a > 0 \\
\cup_{c_i \geq a} E_i \cup (E_1 \cup \dots \cup E_i)^c & \text{if } a \leq 0 \\
\end{cases}$$. Thus $f \in \scriptm$. 

An equivalent definition of a simple function is a measurable function $f: \reals^n \to \reals$ such that $f(\reals^n)$ is a finite set. Indeed, let $\{c_1, \dots, c_d\} = f(\reals^n)$, then $f = \sum_i c_i \chi_{f^{-1}(\{c_i\})}$.

\begin{theorem}
Let $f: \reals^n \to [0, \infty)$ be lebesque measurable. Then $\exists$ a sequence $\{\phi_k\}_{k\in\naturals}$ of simple functions such that $0 \leq \phi_1 \leq \phi_2 \leq \dots \leq f$, and $\lim_{k\to\infty} \phi_k(x) = f(x)$ for all $x\in \reals^n$. Moreover, if $f$ is bounded on $E \in \scriptm$, then $(phi_k)_{k\in\naturals}$ converges uniformly to $f$ on $E$. 
\end{theorem}
\begin{proof}
Fix $k\in\naturals$, then for each integer $0 \leq j \leq 2^{2k} - 1$, let
$$E^{(k)}_j = f^{-1}((j2^{-k}, (j+1)2^{-k}))$$ and
$$F^{(k)} = f^{-1}((2^k, \infty))$$. 
Then set, 
$$\phi_k = \sum_{j=0}^{2^{2k} - 1} j2^{-k} \chi_{E^{(k)}_j} + 2^k\chi_{F^{(k)}}$$
Then $0  \leq \phi_k \leq f$. Also since, $E_j^{(k)} \subseteq E_{2j}^{(k+1)}$, we have
$$j2^{-k}\chi_{E_j^{(k)}} = 2j2^{-(k+1)} \chi_{E_j^{(k)}} \leq 2j2^{-(k+1)} \chi_{E_j^{(k+1)}}$$
So $\phi_k \leq \phi_{k+1}$. Also, on $(F^{(k)})^c = f^{-1}([0, 2^k])$ we have $0 \leq f(x) - \phi_k(x) \leq 2^{-k}$, $\forall x \in (F^{(k)})^c$. Thus $(\phi_k)_{k\in\naturals}$ converges uniformly to $f$ on sets its bounded on. This implies that $(\phi_k)_{k\in\naturals}$ converges pointwise to $f$.
\end{proof}

\begin{corollary}
For $f: \reals^n \to \reals$ measurable, $\exists (\phi_k)_{k\in\naturals}$ simple functions such that $0 \leq |\phi_1| \leq |\phi_2| \leq \dots \leq |f|$, and $\phi_k \to f$ pointwise and $\phi_k \rightrightarrows f$ on any set where $f$ is bounded.  
\end{corollary}
\begin{proof}
Apply the Theorem to $f_{+} = f\chi_{f^{-1}([0, \infty))}$ and $f_{-} = - f\chi_{f^{-1}((-\infty, 0])}$. 
\end{proof}
\begin{definition}
Given $f, g: \reals^n \to \reals$, we say $f=g$ almost everywhere (a.e.) if $\{x\in \reals^n : f(x) \neq g(x)\}$ is a zero set.
\end{definition}
We will see that $f=g$ a.e. implies their lebesque integrals agree.
\begin{prop} The following both follow from this definition:
\begin{itemize}
	\item If $f: \reals^n \to \reals$ is measurable and $g: \reals^n \to \reals$ satisfies $f = g$ a.e. then $g$ is measurable. 
	\item If $(f_k)_{k\in\naturals}$ is a sequence of measurable functions on $\reals^n$ and $g: \reals^n \to \reals$ satisfies $\lim_{k\to\infty} f_k = g$ a.e. Then $g$ is measurable.
\end{itemize}
\end{prop}

\section{The Lebesque Integral}
Let us define $L^{+}$ as the space of all Lebesque measurable functions $f: \reals^{n} \to [0, \infty)$.
\begin{definition}
	For a simple function $\phi = \sum_{j=1}^{n} a_j \chi_{E_j} \in L^{+}$, we define the lebesque integral of $\phi$ as the quantity: 
	$$\int \phi dm \coloneqq \sum_{j=1}^n a_j m(E_j)$$
\end{definition}
\begin{remark} 
If $m(E_j) = \infty$ and $a_j > 0$ then $\int \phi dm = \infty$. Otherwise, if each $m(E_1), \dots, m(E_n) < \infty$, then $\int \phi dm < \infty$.
\end{remark}

\begin{notation}
If we wish to make the argument of $\phi$ explicit we will write $\int \phi(x) dm(x) \coloneqq \int \phi dm$

If $A \in \scriptm$, then
$$\phi \chi_A = \sum_{j =1}^n a_j \chi_{E_j \cap A} \in L^{+}$$,
so its lebesque integral is defined as above. We write 
$$\int_A \phi dm \coloneqq \int \phi \chi_A dm$$
\end{notation}

\begin{prop}
Let $\phi$ and $\psi$ be simple functions in $L^{+}$: \begin{enumerate}
\item If $c \geq 0$ then $\int c\phi dm = c\int \phi dm$
\item $\int (\phi + \psi)dm = \int \phi dm + \int \psi dm$
\item If $\phi \leq \psi$, then $\int \phi dm \leq \int \psi dm$
\end{enumerate}
\end{prop}
\begin{proof}
\begin{enumerate}
	\item This is immediate from the definition of the lebesque integral.
	\item TODO
	\item TODO
\end{enumerate}
\end{proof}
\begin{definition}
For $f \in L^{+}$, the lebesque integral of $f$ is the quantity
$$\int f dm \coloneqq \sup \{\int \phi dm : 0 \leq \phi \leq f, \phi \text{is simple}\}$$
\end{definition}
\begin{remark}
If $f$ above is simple, the part (c) of the previous proposition implies this new definition agrees with the old one.
\end{remark}

\begin{theorem}
Let $B \in \reals^d$ be a box and suppose $f: \reals^d \to [0, \infty)$ is Riemann integrable over $B$. Then the Riemann integral of $f$ over $B$ is equal to the lebesque integral of $f$ over $B$, i.e.
$$\int_B f = \int_B f dm$$
\end{theorem}
\begin{proof}
TODO
\end{proof}
\begin{prop}
\begin{enumerate}
	\item For $f,g \in L^{+}$, if $f \leq g$ then $\int f dm \leq \int g dm$
	\item For $f \in L^{+}$ and $c > 0$, $\int cf dm = c \int f dm$
\end{enumerate}
\end{prop}
\begin{proof}
\begin{enumerate}
	\item For any simple $\phi$ satisfying $0 \leq \phi \leq f$, we have $0 \leq \phi \leq g$. Thus $\int \phi dm \leq \int g dm \Rightarrow \int f dm \leq \int g dm$
	\item Since this holds for simple functions and since a factor of $c$ can be extracted from the supremum we have $\int c f dm = c \int f dm$
\end{enumerate}
\end{proof}
\begin{theorem}[The Monotone Convergence Theorem]
Let $(f_k)_{k\in \naturals} \subseteq L^{+}$ be a sequence satisfying $f_1 \leq f_2 \leq \dots$. Define
$$f = \lim_{n\to\infty} f_n = \sup f_n$$. 
Then $$\int f dm = \lim_{n\to \infty} \int f_n dm$$
\end{theorem}
\begin{proof}
By the previous proposition $\int f_n dm$ is an increasing sequence, so its limit exists, though it may be infinite. Furthermore, $f_n \leq f$ implies
$$\lim_{n\to \infty} \int f_n dm = \sup \int f_n dm \leq \int f dm$$.
 So it suffices to show the reverse inequality. Fix $\alpha \in (0, 1)$ and let $0 \leq \phi \leq f$ be a simple function. Consider $E_n \coloneqq \{x : f_n(x) \geq \alpha \phi(x)\}$. Then $E_1 \subseteq E_2 \subseteq \dots$ and $\cup_{n=1}^\infty E_n = \reals^d$. Indeed, the first statement is clear and for any $x \in \reals^d$, $f(x) > \alpha \phi(x)$. Thus by the definition of the supremum, $\exists n\in\naturals$ such that $f(x) \geq f_n(x) > \alpha \phi(x)$. Thus $x \in E_n$. Now, 
$$\int f_n dm \geq \int_{E_n} f_n dm \geq \int_{E_n} \alpha \phi dm = \alpha \int_{e_n} \phi dm$$
Suppose $\phi = \sum a_j \chi_{B_j}$, then by continuity from below we have
$$lim_{n\to\infty} \alpha \int_{E_n} \phi dm = \lim_{n\to\infty} \alpha \sum a_j m(B_j \cap E_n) = \alpha \sum a_j m(B_j) = \alpha \int \phi dm$$. 
So
$$\lim_{n\to\infty} \int f_n dm \geq \alpha \int \phi dm$$,
and since $0 \leq \phi \leq f$ was arbitrary, we have
$$\lim_{n\to \infty} \int f_n dm \geq \alpha \int f dm$$. Letting $\alpha \to 1$ completes the proof.
\end{proof}
\begin{remark}
The definition of $\int f dm$ involves the $\sup$ over a most likely uncountable set, so it is in practice hard to compute. The Monotone Convergence Theorem allows us to reduce this to a simple limit, in particular, we know $\forall f \in L^{+}, \exists (\phi_n)_{n\in\naturals}$ of simple functions such that $0 \leq \phi_1 \leq \phi_2 \leq \dots \leq f$ such that $\lim_{n\to\infty} \phi_n = f$
\end{remark}
\begin{theorem}
For $(f_n)_{n\in\naturals} \subseteq L^{+}$, 
$$\int \sum_{n=1}^\infty f_n dm = \sum_{n=1}^{\infty} \int f_n dm$$
\end{theorem}
\begin{proof}
First, we claim $\int f_1 + f_2 dm = \int f_1 dm + \int f_2 dm$. By the above remark and the MCT $\exists (\phi_k)_{k\in\naturals}, (\psi_k)_{k\in\naturals}$ of simple functions whose limit is $f_1$ and $f_2$ respectively. Note that $0 \leq \phi_k + \psi_k \leq f_1 + f_2$ and
$$\lim_{k\to\infty} (\phi_k + \psi_k) = (f_1 + f_2)$$.
Hence by the MCT we have:
$$\int(f_1 + f_2)dm = \lim_{k\to\infty} \int \phi_k + \psi_k dm = \lim_{k\to\infty}(\int \phi_k dm + \int \psi_k dm) = \int f_1dm + \int f_2 dm$$
By an inductive argument, we then have 
$$\int \sum_{n=1}^N f_n dm = \sum_{n=1}^N \int f_n dm$$. 
Then applying the MCT again yields
$$ \sum_{n=1}^\infty \int f_n dm = \lim_{N\to\infty} \sum_{n=1}^N \int f_n dm = \lim_{N\to\infty} \int \sum_{n=1}^N f_n dm = \int \sum_{n=1}^\infty f_n dm$$
\end{proof}
\begin{prop}
If $f \in L^{+}$, then $\int f dm = 0$ iff $f = 0$ a.e.
\end{prop}
\begin{proof}
If $f$ is simple, say $f = \sum a_j \chi_{E_j}$, then $\int f dm = 0 \Leftrightarrow \sum a_j m(E_j) = 0 \Leftrightarrow $ for each $j$ either $a_j = 0$ or $E_j$ is a zero set which is equivalent to $f = 0$ a.e. More generally, suppose $f=0$ a.e. If $0 \leq \phi \leq f$ is simple, then $\phi = 0$ a.e. Thus $\int \phi dm = 0$ and so
$$\int f dm = \sup\{\int \phi dm : 0 \leq \phi \leq f, \phi \text{is simple}\} = \sup \{0: ``''\} = 0$$. 
Conversely, suppose $\int fdm = 0$. Let $E_n = \{x : f(x) > \frac{1}{n}\}$. Then $E_1 \subseteq E_2 \subseteq \dots$ and $E \coloneqq \{x : f(x) > 0\} = \cup_{n=1}^\infty E_n$. If $f \neq 0$ a.e., then $m(E) > 0$, and by continuity from below we have
$$m(E) = \lim_{n\to\infty} m(E_n)$$, and so $\exists n\in\naturals$ such that $m(E_n) > 0$. But then $\phi \coloneqq \frac{1}{n} \chi_{E_n} \leq f$ and so
$$\int f dm \geq \int \phi dm = \frac{1}{n} m(E_n) > 0$$, which is a contradiction. 
\end{proof}
\begin{corollary}
If $f, g \in L^{+}$ satisfy $f = g$ a.e., then $\int f dm = \int g dm$.
\end{corollary}
\begin{corollary}
If $f, g \in L^{+}$ and $f \in L^{+}$ are such that $f_1 \leq f_2 \leq \dots \leq f$ and $f(x) = \sup f_n(x)$ a.e., then $\int fdm = \lim_{n\to\infty} \int f_n dm$
\end{corollary}
\begin{proof}
Let $E = \{x: f(x) = \sup f_n(x)\}$. Then $m(E^c) = 0$ and hence $f = f\chi_E$ and $f_n = f_n\chi_E$ a.e. By the previous corollary and the MCT:
$$\int f dm = \int_E f dm = \lim_{n\to\infty} \int_E f_n dm = \lim_{n\to\infty} \int f_n dm$$.
\end{proof}
\begin{remark}
The hypothesis ``$f_1 \leq f_2 \leq \dots$'' in the MCT cannot be readily discarded. For example, $\chi_{(n, n+1)} \to 0$ but $\int \chi_{(n, n+1)} dm = 1$ and $n\chi_{(0, \frac{1}{n})} \to 0$ but $\int n\chi_{(0, \frac{1}{n})} dm = 1$. These examples demonstrate the area under the graph ``escaping to infinity''.
\end{remark}

\begin{theorem}[Fatau's Lemma]
For any $(f_n)_{n\in\naturals} \subseteq L^{+}$,
$$\int \liminf_{n\to\infty} f_n dm \leq \liminf_{n\to\infty} \int f_n dm$$
\end{theorem}
\begin{proof}
For $N\in\naturals$, $\inf_{n\geq N} f_n \leq f_k$ for all $k\geq N$. Thus by monotonicity of the integral we have
$$\int \inf_{n\geq N} f_n dm \leq \int f_k dm$$
 for all $k \geq N$. Therefore,
 $$\int \inf_{n\geq N} f_n dm \leq \inf_{n\geq N} \int f_n dm$$.
 The MCT then implies $$\int \liminf{n\to\infty} f_n dm = \int \lim_{N\to\infty} \inf_{n\geq N} f_n dm = \lim_{N\to\infty} \int \inf{n\geq N} f_n dm \leq \liminf_{n\to\infty} \int f_n dm$$
\end{proof}
\begin{corollary}
If $(f_n)_{n\in\naturals} \subseteq L^{+}$, $f\in L^{+}$, and $f_n \to f$ a.e. then
$$\int f dm \leq \liminf_{n\to\infty} \int f_n dm$$
\end{corollary}
\begin{proof}
If $f_n \to f$ everywhere, then by Fatau's Lemma we are done. We can achieve this by modifying the $f_n$'s and $f$ on a measure zero set, which does not affect the integrals.
\end{proof}

\section{Integrating $\reals$-valued Functions}
Suppose $f: \reals^d \to \reals$ is lebesque measurable. Then if $E = \{x\in \reals^d : f(x) \geq 0\} = f^{-1}([0, \infty))$, we have $E \in \scriptm$ and so $f_{+} = f\chi_E$ and $f_{-} = -f\chi_{E^c}$ are elements of $L^{+}$ with $f = f_{+} - f_{-}$. We want to define the lebesque integral of $f$ as $\int f_{+} dm - \int f_{-}dm$ but if $\int f_{\pm} dm = \infty$, we cannot make sense of ``$\infty - \infty$''. However, note that $|f| = f_{+} + f_{-} \in L^{+}$. If $\int |f|dm < \infty$, then $\int f_{\pm}dm < \infty$ and so we can reconcile $\int f_{+}dm - \int f_{-}dm$.

\begin{definition}
We say a Lebesque measurable function $f: \reals^d \to \reals$ is Lebesque integrable if $\int |f| dm < \infty$. In this case, we define the Lebesque integral of $f$ to be the quantity $\int f dm \coloneqq \int f_{+} dm - \int f_{-} dm$. We denote by $L^1(\reals^d, m)$ the set of Lebesque integrable functions.
\end{definition}
\begin{prop}
If $f\in L^1(m)$, then $|\int f dm| \leq \int |f|dm$.
\end{prop}
\begin{proof}
$|\int f dm| = | \int f_{+}dm - \int f_{-}dm| \leq \int f_{+}dm + \int f_{-}dm = \int |f| dm$
\end{proof}

\begin{prop}
Let $f, g \in L^1(m)$, the following are equivalent:
\begin{enumerate}
	\item $\int_E f dm = \int_E g dm$ for all $E \in \scriptm$
	\item $\int |f - g| dm = 0$
	\item $f = g$ a.e.
\end{enumerate}
\end{prop}
\begin{proof} The proofs for each direction are as follows:
\begin{enumerate}
	\item[(i)$\Rightarrow$ (ii):] We have $\int |f -g| dm = \int (f - g)_+ dm + \int (f-g)_- dm$. Let
	$$E_{+} = \{x : f(x) \geq g(x)\}$$
	and $$E_{-} = \{x : f(x) \leq g(x)\}$$
	Then $(f-g)_{\pm} = (f-g)\chi_{E_\pm}$. Thus
	$$\int (f -g)_\pm dm = \int (f-g)\chi_{E_\pm} = \int_{E_\pm} (f-g) dm = 0$$. So $\int |f - g| dm = 0$ 
	\item[(ii)$\Rightarrow$ (iii)] $\int |f - g| dm \Rightarrow |f - g| = 0 a.e. \Rightarrow f = g a.e.$
	\item[(iii)$\Rightarrow$ (i)] $f = g$ a.e $\Rightarrow f\chi_E = g\chi_E$ a.e. $\forall E \in \scriptm \Rightarrow$ (i).
\end{enumerate}
\end{proof}
\begin{remark}
This proposition tells us that with respect to integration, modifying functions on measure zero sets has no effect. Hence, since $\int |f| dm < \infty \Rightarrow f$ is finite a.e., we can treat every $f \in L^1(m)$ as finitely valued everywhere by setting it to, say, 0 wherever it was originally infinitely valued. In fact, we can go further: we can redefine $L^1(m)$ as the set of equivalence classes of Lebesque integrable functions under the equivalence realtion $f ~ g \Leftrightarrow f =g$ a.e.

\textbf{Advantage:} In this case, for $f,g\in L^1(m)$ we have $\int |f -g | dm = 0$ iff $f = g$ (as equivalence classes). Consequently, $L^1(m)$ is a metric space with metric $d(f, g) \coloneqq \int|f-g| dm$. 

\textbf{Disadvantage:} It no longer makes sense to discuss the value of $f\in L^1(m)$ at a point, since up to a.e. equivalence, the value can be anything.
\end{remark}
\begin{theorem}[Dominated Convergence Theorem]
Let $(f_n)_{n\in\naturals}\subseteq L^1(m)$ be a sequence such that $f_n \to f$ a.e. and $\exists g\in L^1(m)$ such that $|f_n| \leq g$ a.e. for all $n\in\naturals$. 

Then $f \in L^1(m)$ with $\int fdm = \lim_{n\to\infty} \int f_n dm$.
\end{theorem}
\begin{proof}
By an earlier proposition, $f$ is measurable, and $|f| \leq |g|$ a.e. Hence, $\int |f|dm \leq \int |g|dm < \infty$, so $f \in L^1(m)$. Note that $g \pm f \geq 0$ a.e. So by Fatau's Lemma we have
$$\int g dm \pm \int f dm = \int (g \pm f) dm \leq \liminf_{n\to\infty} \int (g\pm f_n) dm = \int g dm + \liminf_{n\to\infty} \int (\pm f) dm$$
So, subtracting $\int g dm$ which is finite from each side, we have $$\int f dm \leq \liminf_{n\to\infty} \int f_n dm$$ and
$$-\int f dm \leq -\limsup_{n\to\infty}\int f_n dm$$
Thus $$\int f dm \leq \liminf_{n\to\infty} \int f_n dm \leq \limsup_{n\to\infty} \int f_n dm \leq \int f dm$$
So all the inequalities are equalities implying $\lim_{n\to\infty} \int f_n dm$ exists and equals $\int f dm$. 
\end{proof}
\begin{theorem}
Suppose $(f_n)_{n\in\naturals}\subseteq L^1(m)$ satisfies $\sum_{n=1}^\infty \int |f_n|dm < \infty$. Then $\sum_{n=1}^\infty f_n$ converges a.e. to a function in $L^1(m)$ and $\int \sum_{n=1}^\infty f_n dm = \sum_{n=1}^\infty \int f_n dm$. 
\end{theorem}
\begin{proof}
By the first theorem after the MCT
$$\int \sum_{n=1}^\infty |f_n| dm = \sum_{n=1}^\infty \int |f_n| dm < \infty$$. 
So $g = \sum_{n=1}^\infty |f_n| \in L^1(m)$. Recall that this implies $g$ is finite a.e. and so the series converges a.e. Since $|\sum_{n=1}^N f_n| \leq \sum_{n=1}^N |f_n| \leq g$, the DCT implies
$$\int \sum_{n=1}^\infty f_n dm = \int \lim_{N\to\infty} \sum_{n=1}^N f_n dm = \lim_{N\to\infty} \int \sum_{n=1}^N f_n dm = \lim_{N\to\infty} \sum_{n=1}^N \int f_n dm = \sum_{n=1}^\infty \int f_n dm$$.
\end{proof}
\begin{theorem}
If $f \in L^1(m)$, for any $\epsilon > 0, \exists \phi \in L^1(m)$ simple such that $\int |f - \phi| dm < \epsilon$.
\end{theorem}
\begin{proof}
Recall we proved $\exists (\phi_k)_{k\in\naturals}$ simple such that $0 \leq |\phi_1| \leq |\phi_2| \leq \dots \leq |f|$ such that $\phi_k \to f$ pointwise. Then $|f - \phi_k| \leq |f| + |\phi_k| \leq 2|f| \in L^1(m)$ and $|f - \phi_k| \to 0$ pointwise. So by the DCT
$$\lim_{k\to\infty} \int |f - \phi_k|dm = \int 0 dm = 0$$. Thus $\forall \epsilon > 0, \exists k\in\naturals$ such that $\phi=\phi_k$ satisfies $\int | f- \phi| dm < \epsilon$.
\end{proof}
\begin{remark}
This says that simple functions are dense in the metric space $(L^1(m), \int|\cdot - \cdot| dm)$.
\end{remark}
\begin{theorem}
If $f\in L^1(m)$ for any $\epsilon > 0$, $\exists g$ continuous and completely supported such that $|f - g| dm < \epsilon$.
\end{theorem}
\begin{proof}
Step 1
Suppose $f = \chi_E$ for $E\in \scriptm$ with $m(E) = \int \chi_E dm < \infty$. By continuity from below,
$$m(E) = \lim_{n\to\infty} m(E \cap B(0, n))$$. So $\exists n\in \naturals$ such that $m(E\cap B(0, n)) > m(E) - \frac{\epsilon}{2}$. Denote $E_0 \coloneqq E \cap B(0, n)$. Observe that
$$\int | \chi_{E_0} - \chi_E | dm = \int \chi_{E\setminus E_0} dm = m(E \setminus E_0) = m(E) - m(E_0) < m(E) - m(E) + \frac{\epsilon}{2} = \frac{\epsilon}{2}$$
Now since $E_0 \in \scriptm$, $\exists$ a $G_\delta$ set and an $F_\sigma$ set, $G = \cap_{i=1}^\infty U_i$, where $U_i$ are open and $F = \cup_{j=1}^\infty V_j$ where $V_j$ are closed, such that $m(G) = m(F) = m(E_0)$ and $F \subseteq E_0 \subseteq G$. Since $E_0$ is bounded, our original construction ensured each $U_i$ is bounded. By continuity from above and below respectively, we have
$$m(E_0) = \lim_{I\to\infty} m(\cap_{i=1}^I U_i)$$
$$m(E_0) = \lim_{J\to\infty} m(\cup_{j=1}^J V_j)$$ So $\exists I, J \in\naturals$ such that $m(\cap_{i=1}^I U_i) < m(E_0) + \frac{\epsilon}{4}$ and $m(\cup_{i=1}^J V_j) > m(E_0) - \frac{\epsilon}{4}$. Set $U = \cap_{i=1}^I U_i$ and $V = \cup_{j=1}^J V_j$. Then $V \subseteq E_0 \subseteq U$ and
$$m(U\setminus V) = m(U) - m(V) < m(E_0) + \frac{\epsilon}{4} - m(E_0) + \frac{\epsilon}{4} = \frac{\epsilon}{2}$$. Define
$$g(x) = \frac{d(x, U^c)}{d(x, U^c) + d(x, V)}$$
 where for a set $S$, $d(x, S) = \inf\{d(x,y) : y\in S\}$. It follows that $0\leq g \leq 1$, $g|_{U^c} = 0$, $g|_V = 1$, and $g$ is continuous. Then
 $$\int |g - \chi_{E_0}| dm \leq \int \chi_{U\setminus V} dm = m(U\setminus V) = \frac{\epsilon}{2}$$. This in combination with the first integral inequality yields $\int |\chi_E - g| dm < \epsilon$.

Step 2
Let $f\in L^1(m)$ be arbitrarry. By the previous theorem, $\exists \phi \in L^1(m)$ simple such that $\int |f - \phi| dm < \frac{\epsilon}{2}$. Suppose $\phi = \sum_{j=1}^N a_j \chi_{E_j}$. Then applying Step 1, for each $j$ we find $g_j$ continuous with compact support and
$$\int | g_j - \chi_{E_j}| dm < \frac{\epsilon}{2N|a_j|}$$
 Then $g  = \sum_{j=1}^N a_j g_j$ is still continuous with compact support, and
 $$\int |\phi - g| dm \leq \sum_{j=1}^N |a_j| \int |\chi_{E_j} - g_j| dm < \sum_{j=1}^N |a_j| \frac{\epsilon}{2N|a_j|} = \frac{\epsilon}{2}$$
 Thus
 $$\int |f - g| dm \leq \int |f - \phi| dm + \int |\phi - g| dm < \frac{\epsilon}{2} + \frac{\epsilon}{2} = \epsilon$$
\end{proof}
\begin{remark}
This theorem says that continuous, compactly supported functions are dense in the metric space $(L^1(m), \int |\cdot - \cdot| dm)$
\end{remark}

\begin{definition}
For $f \in L^1(m)$ define its $L^1$-norm by $||f||_1 \coloneqq \int |f| dm$. Note that $|| \cdot ||_1$ is a valid norm:
\begin{enumerate}
	\item $||cf||_1 = \int |c||f|dm = |c|\int|f|dm = |c|||f||_1$
	\item $||f||_1 = 0 \Leftrightarrow \int |f|dm = 0 \Leftrightarrow f= 0$ a.e. $\Leftrightarrow f = 0$ in $L^1(m)$
	\item $||f + g||_1 = \int|f+g|dm \leq \int |f| + |g| dm = ||f||_1 + ||g||_1$
\end{enumerate}
\end{definition}
\begin{lemma}
Let $(V, ||\cdot||)$ be a normed vector space. Then it is complete iff every absolutely convergent series converges.
\end{lemma}
\begin{proof}
\begin{enumerate}
	\item[($\Rightarrow$)] Suppose $(V, ||\cdot||)$ is complete. Let $(x_n)_{n\in\naturals}\subseteq V$ such that the series $\sum_{n=1}^\infty ||x_n||$ converges. Then for $\epsilon > 0$, $\exists N_0 \in\naturals$ such that
	$$\sum_{n=N_0}^\infty ||x_n|| < \epsilon$$
	Consequently, $\forall N, M \geq N_0, N\leq M$ we have,
	$$||\sum_{n=1}^N x_n - \sum_{n=1}^M x_n || = ||\sum_{n=N+1}^M x_n|| \leq \sum_{n=N+1}^M ||x_n|| \leq \sum_{n=N_0}^M ||x_n|| < \epsilon$$. That is $(\sum_{n=1}^N x_n)_{N\in\naturals}$ is a Cauchy sequence, hence it converges since $V$ is complete. So
	$$\lim_{N\to\infty} \sum_{n=1}^N x_n = \sum_{n=1}^\infty x_n$$
	\item[$\Leftarrow$] Let $(x_n)_{n\in\naturals}$ be a Cauchy sequence. It suffices to show that it has a convergent subsequence. We can find increasing integers $n_1 < n_2 < \dots$ such that $\forall n, m \geq n_k$
	$$||x_n - x_m || < 2^{-k}$$ 
	Now define $y_1 = x_{n_1}$, and $y_k = x_{n_k} - x_{n_{k=1}} \forall k \geq 2$. Observe that
	$$\sum_{k=1}^K y_k = x_{n_1} + \sum_{k=2}^K (x_{n_k} - x_{n_{k-1}}) = x_{n_K}$$
	Now,
	$$\sum_{k=1}^\infty ||y_k|| \leq ||y_1|| + \sum_{k=2}^\infty 2^{-(k - 1)} = ||y_1|| + 1 < \infty$$
	Thus by our hypothesis, $\sum_{k=1}^\infty y_k converges$. This means $$(\sum_{k=1}^K y_k)_{K\in\naturals} = (x_{n_k})_{k\in\naturals}$$ converges. 
	Thus $(x_n)_{n\in\naturals}$ converges and so $(V, ||\cdot||)$ is complete. 
\end{enumerate}
\end{proof}
\begin{theorem}
$L^1(m)$ equipped with the $L^1$-norm is a Banach space, a complete normed vector space.
\end{theorem}
\begin{proof}
We know $L^1(m)$ is a vector space over $\reals$, so it remains to show it is complete. By the Lemma, it suffices to consider absolutely convergent series. Let $(f_n)_{n\in\naturals}\subseteq L^1(m)$ be absolutely convergent.
 $$B\coloneqq \sum_{n=1}^\infty ||f_n||_1 < \infty$$
 Define $G = \sum_{i=1}^\infty |f_n|$ and $G_N = \sum_{n=1}^N |f_n|$. Then by the MCT
 $$\int G dm = \lim_{N\to\infty} \int G_N \leq B < \infty$$ 
 So $G\in L^1(m)$ and in particular $G < \infty$ a.e. Consequently, $\sum_{n=1}^\infty f_n$ converges a.e. Set $F = \sum_{n=1}^\infty f_n$. So that $|F| \leq G$ and have $F \in L^1(m)$. Moreover, for each $N\in \naturals$
 $$|F - \sum_{n=1}^\infty f_n| \leq |F| + \sum_{n=1}^N |f_n| \leq 2|G| \in L^1(m)$$
 So by the DCT
 $$\lim_{N\to\infty} ||F - \sum_{n=1}^N f_n||_1 = \lim_{N\to\infty} \int |F - \sum_{n=1}^N | dm = \int \lim_{N\to\infty} |F - \sum_{n=1}^N f_n| dm = \int 0 dm = 0$$
 Thus $(\sum_{n=1}^N f_n)_{N\in\naturals}$ converges to $F$ w.r.t $||\cdot||_1$.
\end{proof}
\section{Modes of Convergence}
\begin{definition}
For $f, (f_n)_{n\in\naturals} \subseteq L^1(m)$ we say $f_n \to f$ in $L^1$ if $||f_n - f||_1 \to 0$.
\end{definition}
We have seen that $f_n \to f$ a.e. does not imply $f_n \to f$ in $L^1$:
\begin{itemize}
	\item $f_n = \chi_{(n, n+1)}$
	\item $f_n = n\chi_{(0, \frac{1}{n})}$
	\item $f_n = \frac{1}{n}\chi_{(0, n)}$
\end{itemize}
It is also true that $f_n \to f$ in $L^1$ does not imply $f_n \to f$ a.e.
\begin{example}[Type-writer Sequence]
Define a sequence $(x_n)_{n\in\naturals} \subseteq \rationals \cap [0, 1]$.
$$x_0 = \frac{0}{1}, x_1 = \frac{1}{1}, x_2 = \frac{0}{2}, x_3 = \frac{1}{2}, x_4 = \frac{2}{2}, x_5 = \frac{0}{3}, x_6 = \frac{1}{3}, \dots $$
Let
$$ f_n = \begin{cases}
\chi_{[x_{n-1}, x_n]} & \text{if } x_{n-1} < x_n \\
\chi_{[x_n, x_{n-1}]} & \text{otherwise} \\
\end{cases}$$
Then $||f_n||_1 = \int |f_n| dm = \frac{1}{m}$ for some $m\in \naturals$ where $m\to \infty$ as $n\to \infty$. Thus $f_n\to 0$ in $L^1$. However, $\forall x \in [0, 1]$, there are infinitely many $n\in\naturals$ such that $f_n(x) = 1$. So $f_n\not\to 0$ a.e.
\end{example}
\begin{definition}
For $f, (f_n)_{n\in\naturals}\subseteq L^1(m)$, we say $f_n$ converges to $f$ in measure if $\forall \epsilon > 0$,
$$\lim_{n\to\infty} m(\{x\in \reals^d : |f_n(x) - f(x)| > \epsilon\}) = 0$$
\end{definition}
\begin{prop}
If $f_n\to f$ in $L^1$, then $f_n \to f$ in measure.
\end{prop}
\begin{proof}
Define 
$$E_{n, \epsilon} \coloneqq \{x\in\reals^d : |f_n(x) - f(x)| \geq \epsilon\}$$. 
Then
$$\int |f_n - f| dm \geq \int_{E_{n, \epsilon}} |f_n - f| dm \geq \int_{E_{n,\epsilon}} \epsilon dm = \epsilon m(E_{n,\epsilon})$$
Thus,
$$m(E_{n,\epsilon}) \leq \frac{1}{\epsilon} \int |f_n - f| dm$$.
Therefore,
$$\lim_{n\to\infty} m(E_{n,\epsilon}) \leq \frac{1}{\epsilon} \lim_{n\to\infty} ||f_n - f||_1 = 0$$
\end{proof}
\begin{theorem}
Suppose $f_n \to f$ in measure. Then there exists a subsequence $(f_{n_k})_{k\in\naturals}$ such that $f_{n_k}\to f$ a.e.
\end{theorem}
\begin{proof}
For each $k\in\naturals$, choose $n_k$ such that if 
$$E_k \coloneqq \{x: |f_{n_k} - f| \geq 2^{-k}\}$$
Then $m(E_k) \leq 2^{-k}$. Since $\sum m(E_k) \leq \sum 2^{-k} < \infty$. The Borel-Cantelli Lemma implies
$$S\coloneqq \cap_{N\in\naturals} \cup_{k\geq N} E_k, m(S) = 0$$
Notice that
$$S^c = \cup_{N\in\naturals} \cap_{k\geq N} \{x : |f_{n_k} - f| < 2^{-k}\}$$
So for $x \in S^c$, there exists $N \in\naturals$ such that $\forall k\geq N, |f_{n_k}(x) - f(x)| < 2^{-k}$. Thus,
$$\lim_{k\to\infty} |f_{n_k}(x) - f(x)| \leq \lim_{n\to\infty} 2^{-k} = 0$$
That is $f_{n_k}\to f$ on $S^c$. Since $S$ is a zero set, this implies $f_{n_k} \to f$ a.e.
\end{proof}
\begin{theorem}[Egorov's Theorem]
Let $E\in\scriptm$ be such that $m(E) < \infty$. Suppose $f_n \to f$ a.e. on $E$. Then $\forall \epsilon > 0$ there exists an $F \subseteq E$ such that $m(F) < \epsilon$ and $f_n\to f$ uniformly on $E\setminus F$.
\end{theorem}
\begin{proof}
By modifying $f_n, f$ we may assume $f_n \to f$ everywhere on $E$. Then for each $n, k \in \naturals$, define
$$ E_n(k) \coloneqq \cup_{j = n}^\infty \{x: |f_j(x) - f(x)| \geq \frac{1}{k}\}$$
Then for fixed $k$, 
$$ E_1(k) \supseteq E_2(k) \supseteq \dots$$
and $\cap_{n = 1}^\infty E_n(k) = \O$. Since $x$ is in this intersection iff $\forall n\in\naturals \exists j \geq N$ such that $|f_j(x) - f(x)| \geq \frac{1}{k}$, that is $f_n(x) \not\to f(x)$. Now since $m(E_1(k))\leq m(E) < \infty$, continuity from above implies
$$\lim_{n\to\infty} m(E_n(k)) = 0$$
Let $\epsilon > 0$ and let $k\in\naturals$ be arbitrary. Choose $n_k \in \naturals$ large enough such that $m(E_{n_k}(k)) < \epsilon 2^{-k}$. Set
$$E = \cup_{k=1} E_{n_k}(k)$$
So that,
$$ m(E) \leq \sum_{k=1}^\infty m(E_{n_k}(k)) < \sum_{k=1}^\infty \epsilon 2^{-k} = \epsilon$$
Also,
$$ E^c = \cap_{k=1}^\infty E_{n_k}(k)^c = \cap_{k=1}^\infty \cap_{j=n_k}^\infty \{x: |f_j(x) - f(x)| < \frac{1}{k}\}$$
So for $x \in E^c$, we have for all $k\in\naturals$ and $j\geq n_k, |f_j(x) - f(x)| < \frac{1}{k}$. Thus for all $\delta > 0$, letting $\frac{1}{k} < \delta$, we have $\forall j\geq n_k, |f_j(x) - f(x)| < \frac{1}{k} < \delta$. That is $f_n \to f$ uniformly on $E^c$. 
\end{proof}
\end{document}